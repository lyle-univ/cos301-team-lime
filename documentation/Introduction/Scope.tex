NavUP is a project proposed by the University of Pretoria’s department of Computer Science. The objective of the system is to provide the user with an interface from which they can select a location(building or classroom). The system will then calculate the fastest route for the user based on user specified restrictions. The system will also provide information on points of interest on campus. The system will be available to staff, students and visitors to the University of Pretoria.

The system will provide users with an easy way to traverse campus and get interesting information on the various points of interests. This will include historical buildings, envents and activities. Users will be able to set restrictions on the navigation part of the system, for example avoiding pedestrian traffic or accessibility for disabled people. There will also be an option to save locations and routes for future use by the user.

The system will determine the users location both indoors and outdoors through various ways like wifi signal strength, gps location and crowd sourcing. This will allow users to connect to the system and provide them with the necessary information to navigate campus. This is benefitial to new students and visitors that do not know the layout of campus yet. This will also provide users with fast routes to their location when time is of the essence by avoiding pedestrian traffic and activities/events.
