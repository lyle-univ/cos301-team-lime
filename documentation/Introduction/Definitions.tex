\begin{center}
    \begin{tabular}{| l | p{8cm} |}
    \hline
    \textbf{Term} & \textbf{Definition} \\ \hline
	NavUP & The navigation system that is proposed in this document. \\ \hline
	User & The people that will be using the system. \\ \hline
	Guest User & The people that will use the system, but will only have limited functunality. \\ \hline
	Admin/Administrator & Person given permission for managing and controling the system. \\ \hline
	Location & A precise point on a map. \\ \hline
	Venue & A building name or a room within the building itself \\ \hline
	Points of interest & Various locations that may interest users to visit them. \\ \hline
	Events and Activities & Various events/activities that may take place on the campus. \\ \hline
	Restrictions & User selected preferences when using the navigation system NavUP. \\ \hline
	Network & A system consisting of various interconected computers and hardware. \\ \hline
	Heatmaps & An indication (usually in color) of a congested or populated area. \\ \hline
	Wifi & A facility allowing computers, smartphones, or other devices to connect to the Internet or communicate with one another wirelessly within a particular area. \\ \hline
	Wifi hotspots & Areas where users will be able to connect to the network wirelessly. \\ \hline    
	GPS & GPS, which stands for Global Positioning System, is a radio navigation system that allows land, sea, and airborne users to determine their exact location, velocity, and time 24 hours a day, in all weather conditions, anywhere in the world. \\ \hline
	UI/User Interface & The method of which users will interact with the system. \\ \hline
	GIS & A geographic information system (or GIS) is a system designed to capture, store, manipulate, analyze, manage, and present spatial or geographic data. \\ \hline
	Android & An open-source operating system used for smartphones and tablet computers. \\ \hline
	iOS & An operating system used for mobile devices manufactured by Apple Inc. \\ \hline
    \end{tabular}
\end{center}
