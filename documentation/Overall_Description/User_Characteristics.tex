The bulk of users will consist of mobile users. Mobile users will consist of guests with the age group ranging from primary school children to students, alumini and lecturers. Basic competence with smartphones is assumed in addition to navigation skills with map based apps such as google maps.

The web based side of the system will consist of administrators, who should be able to ammend, capture and delete information from the database, thus some knowledge of database management is required. The web based interface will also serve researchers who will be able to analyse data. It is assumed that the researchers have the appropriate skills to conduct their research, this includes domain specific knowledge and basic skills in querying a database.

Lastly the system aims to cater for the disabled. The disabilities which will be accomodated have not yet been fully clarified. But the same level of competence is expected from the disabled as from the general mobile user population to make use of the navigation features. This system is mostly geared toward the physically disabled in order to find the most accomodating routes to classes and buildings.
