The disability user requirements outlines most of the base user requirements.The aim of the disabled navigation is to accommodate disabled user be it students, employees or visitors to successfully use the system.

\FuncReq 
{log-in }
{The user sign-up or must enter their user-name and password to gain access to the system. }
 {N/A}
 {User is logged in or is told that their credentials are incorrect}
     \\
    \textbf{Actor system interaction model: User Sign In and Sign Up }\\
    \begin{tabular}{ | p{6cm} | p{6cm} |}
    \hline
    Actor: Disabled User & System: NavUp \\ \hline
     & 0. The mobile application displays a login screen in which the user must enter their login credentials.\\ \hline
    1. On login, the system displays a menu from which options can be chosen. &\\ \hline   
    \end{tabular}
\\
\bigskip

\FuncReq
{Select voice-assistance }
{The user can select the voice assistance function so system may use voice-command instead of manually selecting the system functions.}
{User must be logged in as a disabled user}
{Interaction between the application and user occur in a verbal manner}
    \\
    \textbf{Actor system interaction model: Voice functionality }\\
    \begin{tabular}{ | p{6cm} | p{6cm} |}
    \hline
    Actor: Disabled User & System: NavUp \\ \hline
     & 0. The mobile app displays an option to choose a voice command function to communicate with the user.\\ \hline
    1. The user selects to employ the functionality. & 2. The application communicates to the user by use of audible commands as well as a visual representation.\\ \hline
    3. The user may respond in the form of a voice command or by using the touch interface on the mobile device. & \\ \hline
    
    \end{tabular}
\\
\bigskip

\FuncReq
{Color vision deficiency check box}
{The user is able to adjust the color displayed by the system in the case of a user with color vision deficiency.}
{User must be logged in as a disabled user}
{Visual aspects of the system are displayed in a format that is visible to the user.}
    \\
    \textbf{Actor system interaction model: Color Adjustment}\\
    \begin{tabular}{ | p{6cm} | p{6cm} |}
    \hline
    Actor: Disabled User & System: NavUp \\ \hline
     & 0. The mobile app displays an option to choose a contrast in colors that is suitable for the user.\\ \hline
    1. The user selects their preferred view. & 2. The visuals of the application are displayed in the contrast selected by the user.\\ \hline   
    \end{tabular}
\\
\bigskip

\FuncReq
{View the map of campus}
{This function is the same as in the base user requirements but also accommodates the vision deficiency in terms of the colors used.}
{User must be logged in as a disabled user}
{Visual aspects of the system are displayed in a format that is visible to the user.}

\FuncReq
{Search for a location/venue on campus to be displayed on the map}
{The user needs to be able to search for a location/venue as stated in the base requirements.The system will display on the map if found or alert the user via voice-command or alert message if not found.}
{User must be logged in as a disabled user}
{Interaction between the application and user occur in a verbal manner}

\FuncReq
{Find the fastest route between two points that are wheel chair friendly.}%This functional requirement will include finding the start point by gps or by manual input from the user
{ This requirement is the same as the base user requirements but includes the voice-command and also accommodates vision deficiency.}
{User must be logged in as a disabled user}
{Interaction between the application and user occur in a verbal manner or a visual manner as determined by the user.}

\FuncReq
{View various points of interest}
{This requirement is the same as the base user requirements but includes the voice-command and also accommodates vision deficiency.}
{User must be logged in as a disabled user}
{Interaction between the application and user occur in a verbal manner or a visual manner as determined by the user.}

\FuncReq
{View any current events or activities happening on campus}
{This requirement is the same as the base user requirements but includes the voice-command and also accommodates vision deficiency.}
{User must be logged in as a disabled user}
{Interaction between the application and user occur in a verbal manner or a visual manner as determined by the user.}

\FuncReq
{Emergency button }
{The user can use this special feature to alert the system in the event of an emergency so that nearest wheel chair friendly route to exit point for the disabled user can be displayed.The feature uses the current location of the user to find the fastest wheel chair friendly route.}
{User must be logged in as a disabled user}
{Emergency personnel are sent to assist the user.}
    \\
    \textbf{Actor system interaction model: Emergency Button}\\
    \begin{tabular}{ | p{6cm} | p{6cm} |}
    \hline
    Actor: Disabled User & System: NavUp \\ \hline
     & 0. The mobile application will display an emergency button or recognise an emergence use-phrase.\\ \hline
    1. The user invokes the use of the emergency button or phrase. & 2. The emergency personnel are notified and are provided with the location of the user.\\ \hline   
    \end{tabular}
\\
\bigskip
