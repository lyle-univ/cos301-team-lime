\subsubsection{User Interface}
The user interface for NavUP application will be specifically designed with users in mind, allowing them to have the best and easiest user interface possible. The aim is to make the user interface easy to navigate so that users spend less time trying to figure out how to use it.
\par
\bigskip
\noindent
We will offer a menu service to allow users of different levels to access their respective sections with ease, these sections would be Student, Lecturer and Guest. If the user chooses Student or Lecturer then possible registration and login screens may be necessary. There will also be a menu service after each user has selected their particular section so as to allow the user to choose their current use of the application. 

\subsubsection{Hardware Interface}
The hardware interface to support the NavUP application would require a smart-phone on which the application will be downloaded and run on. The smart-phone would have to meet the standards of the operating system that the application is going to be developed on,as well as a colour screen with a resolution to match that specified later. 

\subsubsection{Software Interface}
The NavUP application would be designed to run on Android and IOS devices, this is to be confirmed at a later stage. Most devices with this platform feature touchscreens that would allow the application to run at its best. Information can be entered using the on-screen keyboard and the pre-existing touch and drag approach could be used to navigate the map interface. 

\subsubsection{Communications Interface}
The application would mainly use WiFi connections, GPS system and cellphone towers to triangulate and estimate the position/s of the user/s. Information would also have to be collected from the user/s, to allow this, any form of data communications would be allowed, although in essence this would be done mainly through a WiFi or cellphone data connection. 