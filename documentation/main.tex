\documentclass{article}
\usepackage{hyperref}
\hypersetup{
    colorlinks,
    citecolor=black,
    filecolor=black,
    linkcolor=black,
    urlcolor=black
}
\begin{document}
\title{NavUp Requirements specification}
\maketitle
\newpage
{\let\cleardoublepage\clearpage 
\maketitle
\tableofcontents
}
\section{Introduction}
This sectiongives a description and an overview of all the information in the SRS document. The purpose of the system will be provided including additional information of the system.	 
	\subsection{Purpose}
	This document serves to describe the requirements of the NavUP system. This includes the overall features of the system, functionality, interfaces, constraints and integration. The purpose of this document is intended as a communication tool between developers and client to reach consensus on what the client's requirements are for the system. After requirements elicitation and final draft of the SRS is complete, the client should sign off on the document if the client agrees with the specification. After sign off, this document shall be used as a reference material for developers when creating the system.

This document serves to explore the requirements of the NavUP system. This is including the overall features of the system,functionality, interfaces, constraints and integration. The purpose of this document is intended to be proposed to the client for their review and approval aswell as a reference point to developers when creating the system.
	\subsection{Scope}
	NavUP is a project proposed by the University of Pretoria’s department of Computer Science. The objective of the system is to provide the user with an interface from which they can select a location(building or classroom). The system will then calculate the fastest route for the user based on user specified restrictions. The system will also provide information on points of interest on campus. The system will be available to staff, students and visitors to the University of Pretoria.

The system will provide users with an easy way to traverse campus and get interesting information on the various points of interests. This will include historical buildings, envents and activities. Users will be able to set restrictions on the navigation part of the system, for example avoiding pedestrian traffic or accessibility for disabled people. There will also be an option to save locations and routes for future use by the user.

The system will determine the users location both indoors and outdoors through various ways like wifi signal strength, gps location and crowd sourcing. This will allow users to connect to the system and provide them with the necessary information to navigate campus. This is benefitial to new students and visitors that do not know the layout of campus yet. This will also provide users with fast routes to their location when time is of the essence by avoiding pedestrian traffic and activities/events.

NavUP is a project proposed by the University of Pretoria’s department of Computer Science. The objective of the system is to provide the user with an interface from which they can select a location(building or classroom). The system will then calculate the fastest route for the user based on user specified restrictions. The system will also provide information on points of interest on campus. The system will be available to staff, students and visitors to the University of Pretoria.

The system will provide users with an easy way to traverse campus and get interesting information on the various points of interests. This will include historical buildings, envents and activities. Users will be able to set restrictions on the navigation part of the system, for example avoiding pedestrian traffic or accessibility for disabled people. There will also be an option to save locations and routes for future use by the user.

The system will determine the users location both indoors and outdoors through various ways like wifi signal strength, gps location and crowd sourcing. This will allow users to connect to the system and provide them with the necessary information to navigate campus. This is benefitial to new students and visitors that do not know the layout of campus yet. This will also provide users with fast routes to their location when time is of the essence by avoiding pedestrian traffic and activities/events.
	\subsection{Definitions, Acronyms and Abbreviations}
	\begin{center}
    \begin{tabular}{| l | p{7cm} |}
    \hline
    \textbf{Term} & \textbf{Definition} \\ \hline
	NavUP & The navigation system that is proposed in this document. \\ \hline
	User & The people that will be using the system. \\ \hline
	Guest User & The people that will use the system, but will only have limited functunality. \\ \hline
	Admin/Administrator & Person given permission for managing and controling the system. \\ \hline
	Location & A precise point on a map. \\ \hline
	Venue & A building name or a room within the building itself \\ \hline
	Points of interest & Various locations that may interest users to visit them. \\ \hline
	Events and Activities & Various events/activities that may take place on the campus. \\ \hline
	Restrictions & User selected preferences when using the navigation system NavUP. \\ \hline
	Network & A system consisting of various interconected computers and hardware. \\ \hline
	Heatmaps & An indication (usually in color) of a congested or populated area. \\ \hline
	Wifi & A facility allowing computers, smartphones, or other devices to connect to the Internet or communicate with one another wirelessly within a particular area. \\ \hline
	Wifi hotspots & Areas where users will be able to connect to the network wirelessly. \\ \hline    
	GPS & GPS, which stands for Global Positioning System, is a radio navigation system that allows land, sea, and airborne users to determine their exact location, velocity, and time 24 hours a day, in all weather conditions, anywhere in the world. \\ \hline
	UI/User Interface & The method of which users will interact with the system. \\ \hline
	GIS & A geographic information system (or GIS) is a system designed to capture, store, manipulate, analyze, manage, and present spatial or geographic data. \\ \hline
	Android & An open-source operating system used for smartphones and tablet computers. \\ \hline
	iOS & An operating system used for mobile devices manufactured by Apple Inc. \\ \hline
  CRUD & Shorthand for create, read, update and delete. \\ \hline
    \end{tabular}
\end{center}

	
	\subsection{References}
	Bibliography: 
Kung, D.C. (2013) Object-oriented software engineering: An agile unified methodology. 2013th edn. New York: McGraw Hill Higher Education. 
In-line Citation: 
(Kung, 2013) 
	
	\subsection{Overview}
	\input{Introduction/Overview}
	
\section{Overall Description}
	\subsection{Product Perspective}
	\subsubsection{System Interfaces}
\subsubsection{User Interfaces}
\subsubsection{Hardware Interfaces}
\subsubsection{Software Interfaces}
\subsubsection{Communications Interfaces}
\subsubsection{Memory}
\subsubsection{Operations}
\subsubsection{Site Adaptation Requirements}


	\subsection{Product Functions}
	Products function
\par
\bigskip
\noindent
The products function enables the user to enter their personal information and interests to be used in the functionality of the system.The system enables user's to find a route based on their selected destination, preferences and interests using they personal profile.
\par
\bigskip
\noindent
The user's details will be used in tracking the user for various activities such as the compete in location and movement activities.Their personal interests are to be used in new information provided in the event of passing a point of interest, event taking place in the system or route provided to personal preferences.
\par
\bigskip
\noindent
The result of the route should be viewed in the form of a map.The user's location , designated destination,  pedestrian traffic and new information should be also visible whilst moving towards the designated area.

	
	\subsection{User Characteristics}
	The bulk of users will consist of mobile users. Mobile users will consist of guests with the age group ranging from primary school children to students, alumini and lecturers. Basic competence with smartphones is assumed in addition to navigation skills with map based apps such as google maps.

The web based side of the system will consist of administrators, who should be able to ammend, capture and delete information from the database, thus some knowledge of database management is required. The web based interface will also serve researchers who will be able to analyse data. It is assumed that the researchers have the appropriate skills to conduct their research, this includes domain specific knowledge and basic skills in querying a database.

	
	\subsection{Constraints}
	The customer did not voice any strict constraints.

	
	\subsection{Assumptions and Dependencies}
	Factors that could affect the requirements include compliance with the Protection of Private Information Act of 2013. The POPI act is applicable since user location data gets captured and assigned to a persistent ID and then stored in a database. This means that compliance requires certain additions to functional requirements such as asking the data-subject for consent when the app is opened for the very first time and allowing withdrawal of consent and destruction of identifiable data within a reasonable time. It could also potentially restrict certain functional requirements such as surveilance and analysis.
\par
\bigskip
\noindent
For the system to function as expected, we must assume that there will be sufficient wifi and GPS coverage and the sensors, when integrated, will provide sufficient accuracy to serve its purpuse. We also assume that map data will be available with a sufficient degree of accuracy and completeness to satisfy the requirements of the system.

	
\section{Specific Requirements}
	\subsection{External Interface Requirements}
	\subsubsection{User Interface}
The user interface for NavUP application will be specifically designed with users in mind, allowing them to have the best and easiest user interface possible. The aim is to make the user interface easy to navigate so that users spend less time trying to figure out how to use it.

We will offer a menu service to allow users of different levels to access their respective sections with ease, these sections would be Student, Lecturer and Guest. If the user chooses Student or Lecturer then possible registration and login screens may be necessary. There will also be a menu service after each user has selected their particular section so as to allow the user to choose their current use of the application. 

\subsubsection{Hardware Interface}
The hardware interface to support the NavUP application would require a smart-phone on which the application will be downloaded and run on. The smart-phone would have to meet the standards of the operating system that the application is going to be developed on,as well as a colour screen with a resolution to match that specified later. 

\subsubsection{Software Interface}
The NavUP application would be designed to run on Android and IOS devices, this is to be confirmed at a later stage. Most devices with this platform feature touchscreens that would allow the application to run at its best. Information can be entered using the on-screen keyboard and the pre-existing touch and drag approach could be used to navigate the map interface. 

\subsubsection{Communications Interface}
The application would mainly use WiFi connections, GPS system and cellphone towers to triangulate and estimate the position/s of the user/s. Information would also have to be collected from the user/s, to allow this, any form of data communications would be allowed, although in essence this would be done mainly through a WiFi or cellphone data connection. 
	
	\subsection{Functional Requirements}
	\newcounter{FuncReqSerial}

\newcommand{\FuncReq} [4]{
    \refstepcounter{FuncReqSerial}

    \textbf{Serial}:F\theFuncReqSerial\\
    \textbf{Abstract}: #1\\
    \textbf{Description}: #2\\
    \textbf{Pre-condition}: #3\\
    \textbf{Post-condition}: #4\\
}

	
	\subsection{Performance Requirements}
	\input{Specific_Requirements/Specific_Requirements}
	
	\subsection{Design Constraints}
	No constraints have been voiced by the client

	
	\subsection{Software System Attributes}
	\input{Specific_Requirements/Software_System_Attributes}
	
	\subsection{Other Requirements}
	\input{Specific_Requirements/Other_Requirements}

\end{document}
